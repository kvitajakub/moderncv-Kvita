%% start of file `template.tex'.
%% Copyright 2006-2013 Xavier Danaux (xdanaux@gmail.com).
%
% This work may be distributed and/or modified under the
% conditions of the LaTeX Project Public License version 1.3c,
% available at http://www.latex-project.org/lppl/.


\documentclass[11pt,a4paper,roman]{moderncv}        % possible options include font size ('10pt', '11pt' and '12pt'), paper size ('a4paper', 'letterpaper', 'a5paper', 'legalpaper', 'executivepaper' and 'landscape') and font family ('sans' and 'roman')

% moderncv themes
\moderncvstyle{classic}                           % style options are 'casual' (default), 'classic', 'oldstyle' and 'banking'
\moderncvcolor{blue}                               % color options 'blue' (default), 'orange', 'green', 'red', 'purple', 'grey' and 'black'
%\renewcommand{\familydefault}{\sfdefault}         % to set the default font; use '\sfdefault' for the default sans serif font, '\rmdefault' for the default roman one, or any tex font name
%\nopagenumbers{}                                  % uncomment to suppress automatic page numbering for CVs longer than one page

% character encoding
\usepackage[utf8]{inputenc}          % if you are not using xelatex ou lualatex, replace by the encoding you are using
%\usepackage{CJKutf8}                 % if you need to use CJK to typeset your resume in Chinese, Japanese or Korean

% adjust the page margins
\usepackage[scale=0.85]{geometry}
\setlength{\hintscolumnwidth}{35mm}                % if you want to change the width of the column with the dates
%\setlength{\makecvtitlenamewidth}{10cm}           % for the 'classic' style, if you want to force the width allocated to your name and avoid line breaks. be careful though, the length is normally calculated to avoid any overlap with your personal info; use this at your own typographical risks...

% timeline in dates
\usepackage{moderntimeline}
% Set the scale.
% I go up to 2013 otherwise some late dates overflow on the entry
\tlmaxdates{2010}{2018}
% Set the line width.
% This automatically sets the space under the top label to be 1pt more
\tlwidth{0.8ex}
% Set the labels text size
% \tltext{\tiny}
% http://tex.stackexchange.com/questions/29725/putting-a-timeline-for-dates-in-moderncv
% http://ctan.mirrorcatalogs.com/macros/latex/contrib/moderntimeline/moderntimeline.pdf


\newcommand{\cvdoublecolumn}[2]{%
  \cvline{}{%
    \begin{minipage}[t]{\listdoubleitemmaincolumnwidth}#1\end{minipage}%
    \hfill%
    \begin{minipage}[t]{\listdoubleitemmaincolumnwidth}#2\end{minipage}%
    }%
  \vspace{12pt}   
}

\newcommand{\cvreference}[7]{%
    \textbf{#1}\newline% Name
    \ifthenelse{\equal{#2}{}}{}{\addresssymbol~#2\newline}%
    \ifthenelse{\equal{#3}{}}{}{#3\newline}%
    \ifthenelse{\equal{#4}{}}{}{#4\newline}%
    \ifthenelse{\equal{#5}{}}{}{#5\newline}%
    \ifthenelse{\equal{#6}{}}{}{\emailsymbol~\texttt{#6}\newline}%
    \ifthenelse{\equal{#7}{}}{}{\phonesymbol~#7}}

\newcommand{\googlemap}[1]{\href{https://maps.google.com/maps?&q=#1}{#1}}
\newcommand{\BUT}{\href{http://www.vutbr.cz/en/}{Brno University of Technology}}

% personal data
\name{Jakub}{Kvita}
%\title{Resumé title}                               % optional, remove / comment the line if not wanted
\address{Zauličí 194}{Štramberk 742 66}{The Czech Republic}% optional, remove / comment the line if not wanted; the "postcode city" and and "country" arguments can be omitted or provided empty
\phone[mobile]{\href{tel:+420702973246}{+420~702~973~246}}                   % optional, remove / comment the line if not wanted
%\phone[fixed]{+2~(345)~678~901}                    % optional, remove / comment the line if not wanted
%\phone[fax]{+3~(456)~789~012}                      % optional, remove / comment the line if not wanted
\email{kvitajakub@gmail.com}                               % optional, remove / comment the line if not wanted
\homepage{www.linkedin.com/in/kvitajakub}    % optional, remove / comment the line if not wanted
%\extrainfo{additional information}                 % optional, remove / comment the line if not wanted
\photo[100pt][0.4pt]{../fig/2016.jpg}                 % optional, remove / comment the line if not wanted; '64pt' is the height the picture must be resized to, 0.4pt is the thickness of the frame around it (put it to 0pt for no frame) and 'picture' is the name of the picture file
%\quote{Some quote}                                 % optional, remove / comment the line if not wanted

% to show numerical labels in the bibliography (default is to show no labels); only useful if you make citations in your resume
%\makeatletter
%\renewcommand*{\bibliographyitemlabel}{\@biblabel{\arabic{enumiv}}}
%\makeatother
%\renewcommand*{\bibliographyitemlabel}{[\arabic{enumiv}]}% CONSIDER REPLACING THE ABOVE BY THIS

% bibliography with mutiple entries
%\usepackage{multibib}
%\newcites{book,misc}{{Books},{Others}}
%----------------------------------------------------------------------------------
%            content
%----------------------------------------------------------------------------------
\begin{document}
%\begin{CJK*}{UTF8}{gbsn}                          % to typeset your resume in Chinese using CJK
%-----       resume       ---------------------------------------------------------
\makecvtitle
%\justify
\section{Education}
\tlcventry{2013}{2016}{M.Sc.}{\BUT}{Faculty of Information Technology}{\googlemap{Brno}}{}
\tlcventry{2013}{2014}{M.Sc.}{\href{http://www.southwales.ac.uk/}{University of South Wales}}{Faculty of Computing, Engineering and Science}{\googlemap{Cardiff}}{}
\tlcventry{2010}{2013}{B.Sc.}{\BUT}{Faculty of Information Technology}{\googlemap{Brno}}{}

\section{Experience}
%\subsection{Vocational}
{\tltextstart[base]{\scriptsize}
	\tllabelcventry{2015.7}{2015.8}{2015/7 -- 2015/8}{OpenLab Summer Student}{\href{http://openlab.web.cern.ch/}{CERN}}{\googlemap{Geneve}}{}{I worked in the \emph{Cloud \& Virtualization} team on configuring backends and backups for Openstack Cinder volumes. I used Ceph RBD and TSM clients for backends.}

	\tllabelcventry{2015.1}{2015.6}{2015/2 -- 2015/6}{Quality Assurance Engineer}{\href{www.redhat.com}{Red Hat}}{\googlemap{Brno}}{}{Internship at \emph{REST} team -  we had been creating test cases for the Pulp project with REST API Python tests, Nosetests and communicated with the developers.}
}
\tlcventry{2014}{2015}{Quality Assurance Engineer}{\href{www.redhat.com}{Red Hat}}{\googlemap{Brno}}{}{Internship at \emph{Subscriptions} team - content and SKU testing. Testing subscriptions of customers and CDN content with Python and Redmine+Trac. Team was scattered around the world with members in China, India, US and Europe.}

%Detailed achievements:%
%\begin{itemize}%
%\item Achievement 1;
%\item Achievement 2, with sub-achievements:
% \begin{itemize}%
%  \item Sub-achievement (a);
%  \item Sub-achievement (b), with sub-sub-achievements (don't do this!);
%    \begin{itemize}
%    \item Sub-sub-achievement i;
%   \item Sub-sub-achievement ii;
%   \item Sub-sub-achievement iii;
%   \end{itemize}
% \item Sub-achievement (c);
% \end{itemize}
%item Achievement 3.
%end{itemize}}
%\cventry{year--year}{Job title}{Employer}{City}{}{Description line 1\newline{}Description line 2}
%\subsection{Miscellaneous}
%\cventry{2012}{Forest Keeper}{Lesnicko–dřevařská firma, s.r.o.}{Zubří}{}{Maintaining forests.}

\section{Courses and Certificates}
\cvitem{June 2014}{Computer Vision and Intelligent Computer Systems. Intensive course at \emph{\href{http://en.u-bourgogne.fr/}{University of Burgundy}, France.}}
\cvitem{May 2013}{Cambridge ESOL FCE. \emph{Certificate in english, level C1.}}
\cvitem{January 2013}{Cisco Certified Network Associate. \emph{Routing and Switching.}}
\cvitem{May 2012}{Conflicts and Assertivity. \emph{Soft skills course.}}
\cvitem{January 2012}{Microsoft Certified Technology Specialist: Windows 7, Configuration.}
\cvitem{June 2009}{Driving License.}

\section{Languages}
\cvitemwithcomment{Czech}{native}{}
\cvitemwithcomment{English}{fluent}{Level C1 certificate. Can use language flexibly and effectively.}
\cvitemwithcomment{Slovak}{proficient}{Able to communicate at social, academic and professional level.}
\cvitemwithcomment{Russian}{pre-intermediate}{Can understand and deal with most common situations.}

\section{Skills}
%\cvdoubleitem{category 1}{XXX, YYY, ZZZ}{category 4}{XXX, YYY, ZZZ}
%\cvdoubleitem{category 2}{XXX, YYY, ZZZ}{category 5}{XXX, YYY, ZZZ}
%\cvdoubleitem{category 3}{XXX, YYY, ZZZ}{category 6}{XXX, YYY, ZZZ}
\cvitem{Languages}{Python, Lua, Java, C/C++, Bash, SQL}
\cvitem{Software}{Torch, Git, OpenStack, Ceph, UNIX/Linux, Trac, OpenGL, OpenCV, Matlab}
\cvitem{Concepts}{machine learning, software development, virtualization, quality assurance, rendering, 3D modeling, computer vision, databases}


%\section{Bachelor thesis}
%\cvitem{Title}{\href{https://github.com/kvitajakub/BachelorThesis/blob/master/bakalarka\%20-\%20wis.pdf?raw=true}{\emph{Generator of 3D objects based on L-Systems.}}}
%%\cvitem{Supervisor}{\href{http://www.fit.vutbr.cz/~ivrabel/index.php.en}{Lukáš Vrábel}}
%\cvitem{Description}{Interactive application for generating and viewing 3D models using OpenGL. System is able to model fractals and simulate growth of plants through L-system rules.}

\section{Projects}
%Processor Cooler Simulation
\cventry{OpenMP, OpenMPI}{\href{https://github.com/kvitajakub/coolerSimulation}{Processor Cooler Simulation}}{2015}{\BUT}{}{Parallel simulation of heat diffusion in the cooler mounted on a processor. Two exactly same implementations were created, one with OpenMP library and second using OpenMPI.}

%Raytracing Chess
\cventry{rendering}{\href{https://github.com/kvitajakub/raytracedChess}{Chess Renderer using Ray Tracing}}{2014}{\BUT}{}{The renderer of the chess model based on ray tracing. Simply input the positions of the figures on board. Model will be configured properly and appropriate scene will be rendered.}

%Number Plates Recognition
\cventry{OpenCV}{\href{https://github.com/kvitajakub/numberPlatesRecognition}{Automatic Number Plate Recognizer}}{2014}{\BUT}{}{Computer vision tool, which takes picture, detect licence plates, use classifier to recognize letters and then write them in the image.}

% Video Stabilization
\cventry{OpenCV}{\href{https://github.com/kvitajakub/VideoStabilization}{Video Stabilization}}{2014}{\BUT}{}{Command line tool for stabilizing videos. Two different methods - GrayCoded bitplane matching and KLT tracker and SVD.}

% Lua Interpreter
\cventry{C, interpreters}{\href{https://github.com/kvitajakub/interpreterLua}{Lua Interpreter}}{2011}{\BUT}{}{Command-line interpreter - parse input, check syntax, compile to intermediate code and interpret it. Interpreter can deal with most common features of Lua like variables, cycles, functions,.. Can be used for real programming.}

\section{Qualities}
\cvline{Personal skills}{Strong analytical and complex thinking, focused on solving the problem, keeping schedule, direct communication and self-reflection - I can admit mistakes and be open-minded.}

\cvline{International Experience}{I lived in \googlemap{Cardiff}, where I studied for half a year at university with large international community and shared an apartment with many different foreigners. I also attended several courses at universities abroad. %
\newline 
Work at Red Hat and at CERN prepared me for cooperation within international team and to collaborate everyday with colleagues from different cultures.
}

\section{References}
\cvdoublecolumn{\cvreference{\href{https://www.linkedin.com/in/bukowiec/}{Sebastian Bukowiec}}
    {\emph{CERN} -- Staff member}
	{CERN, Office 31/1-028}
    {CH-1211 Geneva 23, Switzerland}
    {}    
    {\href{mailto:sebastian.bukowiec@cern.ch}{sebastian.bukowiec@cern.ch}}
    {\href{tel:+41227678086}{+41 22 76 78086}}
    }
{\cvreference{\href{https://www.linkedin.com/in/martinkoci}{Martin Kočí}}
    {\emph{Red Hat} -- Quality Engineering Manager}
    {Purkyňova 3080/97B}
    {Brno 612 45, Czech Republic}
    {}
    {\href{mailto:mkoci@redhat.com}{mkoci@redhat.com}}
    {\href{tel:+420702204894}{+420 702 204 894}}
    }    

% Libin Liu (刘俐斌)
\cvdoublecolumn{\cvreference{Liu Libin}
    {\emph{Red Hat} -- Quality Assurance Engineer}
    {9/F, North Tower C, Raycom Infotech Park No.2, Ke Xueyuan Nanlu}
	{Haidian, Beijing 100080, China}
    {}
    {\href{mailto:liliu@redhat.com}{liliu@redhat.com}}
    {\href{tel:+861062608319}{+86 10 62608319}}
    }
{
%\cvreference{\href{https://www.linkedin.com/in/martinkoci}{Martin Kočí}}
%    {\emph{Red Hat} - Quality Engineering Manager}
%    {Purkyňova 3080/97B}
%    {Brno 612 45, Czech Republic}
%    {}
%    {\href{mailto:mkoci@redhat.com}{mkoci@redhat.com}}
%    {\href{tel:+420702204894}{+420 702 204 894}}%
    } 

   

\end{document}

%\section{References}
%\begin{cvcolumns}
%  \cvcolumn{Category 1}{\begin{itemize}\item Person 1\item Person 2\item Person 3\end{itemize}}
%  \cvcolumn{Category 2}{Amongst others:\begin{itemize}\item Person 1, and\item Person 2\end{itemize}(more upon request)}
%  \cvcolumn[0.5]{All the rest \& some more}{\textit{That} person, and \textbf{those} also (all available upon request).}
%\end{cvcolumns}

% Publications from a BibTeX file without multibib
%  for numerical labels: \renewcommand{\bibliographyitemlabel}{\@biblabel{\arabic{enumiv}}}% CONSIDER MERGING WITH PREAMBLE PART
%  to redefine the heading string ("Publications"): \renewcommand{\refname}{Articles}
\nocite{*}
\bibliographystyle{plain}
\bibliography{publications}                        % 'publications' is the name of a BibTeX file

% Publications from a BibTeX file using the multibib package
%\section{Publications}
%\nocitebook{book1,book2}
%\bibliographystylebook{plain}
%\bibliographybook{publications}                   % 'publications' is the name of a BibTeX file
%\nocitemisc{misc1,misc2,misc3}
%\bibliographystylemisc{plain}
%\bibliographymisc{publications}                   % 'publications' is the name of a BibTeX file

\clearpage
%-----       letter       ---------------------------------------------------------
% recipient data
\recipient{Company Recruitment team}{Company, Inc.\\123 somestreet\\some city}
\date{January 01, 1984}
\opening{Dear Sir or Madam,}
\closing{Yours faithfully,}
\enclosure[Attached]{curriculum vit\ae{}}          % use an optional argument to use a string other than "Enclosure", or redefine \enclname
\makelettertitle

Lorem ipsum dolor sit amet, consectetur adipiscing elit. Duis ullamcorper neque sit amet lectus facilisis sed luctus nisl iaculis. Vivamus at neque arcu, sed tempor quam. Curabitur pharetra tincidunt tincidunt. Morbi volutpat feugiat mauris, quis tempor neque vehicula volutpat. Duis tristique justo vel massa fermentum accumsan. Mauris ante elit, feugiat vestibulum tempor eget, eleifend ac ipsum. Donec scelerisque lobortis ipsum eu vestibulum. Pellentesque vel massa at felis accumsan rhoncus.

Suspendisse commodo, massa eu congue tincidunt, elit mauris pellentesque orci, cursus tempor odio nisl euismod augue. Aliquam adipiscing nibh ut odio sodales et pulvinar tortor laoreet. Mauris a accumsan ligula. Class aptent taciti sociosqu ad litora torquent per conubia nostra, per inceptos himenaeos. Suspendisse vulputate sem vehicula ipsum varius nec tempus dui dapibus. Phasellus et est urna, ut auctor erat. Sed tincidunt odio id odio aliquam mattis. Donec sapien nulla, feugiat eget adipiscing sit amet, lacinia ut dolor. Phasellus tincidunt, leo a fringilla consectetur, felis diam aliquam urna, vitae aliquet lectus orci nec velit. Vivamus dapibus varius blandit.

Duis sit amet magna ante, at sodales diam. Aenean consectetur porta risus et sagittis. Ut interdum, enim varius pellentesque tincidunt, magna libero sodales tortor, ut fermentum nunc metus a ante. Vivamus odio leo, tincidunt eu luctus ut, sollicitudin sit amet metus. Nunc sed orci lectus. Ut sodales magna sed velit volutpat sit amet pulvinar diam venenatis.

Albert Einstein discovered that $e=mc^2$ in 1905.

\[ e=\lim_{n \to \infty} \left(1+\frac{1}{n}\right)^n \]

\makeletterclosing

%\clearpage\end{CJK*}                              % if you are typesetting your resume in Chinese using CJK; the \clearpage is required for fancyhdr to work correctly with CJK, though it kills the page numbering by making \lastpage undefined
\end{document}


%% end of file `template.tex'.

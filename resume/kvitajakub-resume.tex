\documentclass{tccv}
\usepackage[english]{babel}

\begin{document}

\part{Jakub Kvita}

\section{Education}

\begin{yearlist}
	
	\item[Computer Science]{2014 -- 2016}
	{Master of Science}
	{Brno University of Technology, Brno, Czechia.}
	
	\item[Computer Science]{2013 -- 2014}
	{Master of Science}
	{University of South Wales, Cardiff, United Kingdom.}
	
	\item[Computer Science]{2010 -- 2013}
	{Bachelor of Science}
	{Brno University of Technology, Brno, Czechia.}
	
\end{yearlist}

\section{Work experience}

\begin{eventlist}

\item{July 2015 -- August 2015}
     {CERN -- European Org. for Nuclear Research}
     {OpenLab Summer Student}

Located at Geneva, Switzerland, I worked in the \emph{Cloud \& Virtualization} team on configuring backends and backups for Openstack Cinder volumes. I used Ceph RBD and TSM clients for backends.

\item{February 2015 -- June 2015}
     {Red Hat, Inc.}
     {Quality Assurance Engineer}

Internship at \emph{REST} team -  we had been creating test cases for the Pulp project with REST API Python tests, Nosetests and communicated with the developers.

\item{June 2014 -- January 2015}
     {Red Hat, Inc.}
     {Quality Assurance Engineer}

Internship at \emph{Subscriptions} team - content and SKU testing. Testing subscriptions of customers and CDN content with Python and Redmine+Trac. Team was scattered around the world with members in China, India, US and Europe.

\end{eventlist}

\section{Courses and Certificates}

\begin{yearlist}
	
	\item{2014}
	{Computer Vision and Intelligent Computer Systems}
	{Intensive course at \emph{University of Burgundy, France.}}
	
	\item{2013}
	{Cambridge ESOL FCE.}
	{\emph{Certificate in English, level C1.}}
	
	\item{2013}
	{Cisco Certified Network Associate}
	{\emph{Routing and Switching.}}
	
	\item{2012}
	{Microsoft Certified Technology Specialist}
	{Windows 7, Configuration.}
	
\end{yearlist}

\personal
[kvitajakub.github.io]
{Zaulici 194, Stramberk 742 66\newline The Czech Republic}
{+420 702 973 246}
{kvitajakub@gmail.com}
{linkedin.com/in/kvitajakub}

\section{Communication skills}

\begin{factlist}
\item{Czech}{Native speaker.}
\item{English}{Fluent. C1 certificate.}
\item{Slovak}{Proficient.}
\item{Russian}{Basic communication skills.}
\end{factlist}

\section{Software skills}

\begin{factlist}

\item{Languages}
     {Python, Lua, Java, C, shell, SQL.}

\item{Software}
     {Torch, Git, OpenStack, Linux, Trac, OpenGL, OpenCV.}

\item{Concepts}
     {Machine learning, quality assurance, computer vision, virtualization.}

\end{factlist}

\section{Projects}

\begin{eventlist}
	
	\item{2016}
	{Masters Thesis}
	{\href{https://github.com/kvitajakub/MasterThesis}{Image Captioning with Recurrent\newline Neural Networks}}
	
	RNN-LSTM models generating text on character level created in Torch. Experiments with initialization of the model by CNN output and bag-of-words vector to create image captions. 
	
	\item{2013}
	{Bachelors Thesis}
	{\href{https://github.com/kvitajakub/LSystemModeller}{Generator of 3D objects based on L-Systems}}
	
	Interactive application for generating and viewing 3D models using OpenGL. System is able to model fractals and simulate growth of plants through L-system rules.
	
	\item{2011}
	{Course project}
	{\href{https://github.com/kvitajakub/interpreterLua}{Lua Interpreter}}
	
	Command-line interpreter - parse input, check syntax, compile to intermediate code and interpret it. Interpreter can deal with most common features of Lua like variables, cycles, functions, etc.
	
\end{eventlist}

\end{document}
